\begin{myfigure}
\psfig{figure=lang-spec.eps,width=6in}
\caption{Compositon of modules \label{algebraic-glue}}
\end{myfigure}
Now that most of the major components have been discussed, we can discuss
how one actually goes about building a non-trivial langauge. We'll do this
by explain how support for ML is provided. Figure \ref{algebraic-glue}
describes how several important functors interact to produce a module which
implements the support for ML code. Each bold arrow represents the use or
import of one module by the other. Any signature between the path of an
import represents an import constraint. 

At the top of the graph is a structure that represents the syntax of general
algebraic languages. The left hand path is a series of module compositions
that map ASDL semantics onto the semantics of an algebraic language. The
right hand builds intrastructure to pretty print the syntax into ML. A the
bottom of the graph we see a functor that merges the syntax and semantics
into a single unit that produces ML code from ASDL specifications. Note that
this graph is not complete and leaves out a few parameters as well as
important sharing constraints, but it accurately describes the high-level
picture of how things are composed.
