\section{About This Document} 
The purpose of this document is to provide a high-level overview of how
\asdlGen is structured. It assumes familiarity with ASDL and Standard ML in
general. It's purpose is to be a ``hackers guide'' and not a user reference
manual. The main goal of this document is to make fixing bugs and adding new
features to \asdlGen a bit easier for programmers reasonably familiar with
Standard ML. All the source code in this document and most of the prose
comes directly for the \asdlGen sources which are annotated with stylized
comments which a tool ({\tt dtangle}) than extracts and converts into \LaTeX
with the help of {\tt noweb}. Those familiar with {\tt noweb} should note
that we are completely reversing the normal way of doing things and deriving
{\tt noweb} documents from source. So one should always edit the sources.

\section{The Big Picture}
\begin{myfigure}
\psfig{figure=big-picture.eps,width=6in}
\caption{Data flow through \asdlGen \label{big-picture}}
\end{myfigure}
Figure \ref{big-picture} is a high-level diagram of how ASDL descriptions
are transformed into source code for different target languages. Each node
in the graph is either an ML structure (solid box), functor (diamond), or
signature constraint (plain text). An arrow indicates the direction of data
flow. Data that passes through a signature means that the data is accessed
through the interface described by the signature. For example if we we're
producing ML code as output, data from the {\tt Semant} structure
constrained by the {\tt SEMANT} signature would be used by the {\tt
mkAlgebraicSemantTranslator} functor whose output would be constrained by
the {\tt SEMANT\_TRANSLATOR} signature. The output then flows through the
{\tt mkTranslateFromTranslator} functor whose output is constrained by the
{\tt TRANSLATE} signature where the type of {\tt output} should be {\tt
Algebraic.module list} and then from there it is passed to the {\tt MLPP}
structure which builds a pretty printed representation of ML source
expressions that are finally passed to the {\tt mkSourceFileTranslate}
functor which pretty prints the value constrained by the {\tt CODE\_PP}
signature.

\asdlGen attempts to deal with the problems of supporting several different
target languages, by separating the semantics of the target language from
the syntax. This encourages code sharing and makes adding new languages that
differ only syntactically easier. Languages are grouped into the following
broad categories:
\begin{description}
\item[Algebraic] - Languages with and algebraic datatypes and
polymorphims. e.g. Standard ML and Haskell
\item[Algol] - Block structured languages in the Algol family with
variant records. e.g. C and Pascal
\item[Object Oriented] - Languages with objects and subtyping e.g. Java and
C++
\end{description}
The ASDL type declarations are converted into the appropriate language level
declarations for a given language family. Since the exact details of each
language vary slightly the translation from ASDL type declarations into a
language category is parametrized by arguments to the a functor that builds
an appropriate translation. There are a few special purpose translations such
as the conversion of ASDL type declarations into HTML documentation.

After converting the ASDL type declarations into a language specific
semantics, a particular kind of abstract syntax tree, it is converted into a
pretty printer specification and then passed to a pretty printer that nicely
formats the resulting code.
 




